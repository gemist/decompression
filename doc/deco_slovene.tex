\documentclass[a4paper]{article}
\usepackage{graphicx}
\usepackage[slovene]{babel}
\usepackage{amsmath}
\usepackage[utf8]{inputenc}
\author {Alan Bizjak}
\title {Uvod v Buhlmanov dekompresijski model}
\date{30.10.2010}
\begin{document}
\maketitle

\section{Uvod}

V potapljaški literaturi je mnogo govora o dekompresiji in
različnih dekompresijskih modelih. Na žalost skoraj nobena
literatura ne govori o samem algoritmu. Med drugim tudi ni nič
napisanega v slovenščini, zato sem se odločil da opišem
Buhlmanov dekompresijski model. Naj omenim samo to da Buhlman ni
``izumil'' modela, temveč je večino samo zbral in izpopolnil
delo drugih kot so John S. Haldane, Robert D. Workman (U.S. Navy) in
Heinz R. Schreiner.

\section{Absorbcija inertnega plina v tkivo na konstatni globini}
Za začetek poglejmo kaj se dogaja z našim telesom na
konstantni globini oziroma pri konstantnem tlaku. Zaradi enostavnosti
prespostavimo da smo se v trenuktu potopilo na globino s tlakom
$P_{alv}$.  V našem telesu je manjši tlak inertnega plina kot
v pljučih ($P_{alv}$). Hitrost spremembe tlaka v tkivu je
sorazmerna razliki tlakov v pljučih in v telesu. Z enačbo to
zapišemo kot
\begin{equation}
  \frac{\mathrm{d} P(t)}{\mathrm{d} t}=k\left(P_{alv}(t)-P(t)\right)
\end{equation} 
\begin{itemize}
\item{$P(t)$ - parcialni tlak inertnega plina v tkivu}
\item{$P_{alv}(t)$ - parcialni tlak inertnega plina v dihalni mešanici oziroma v pljučih (alveolih). Le ta se lahko s časom spreminja, zato je napisana tudi časovna odvisnost.}
\item{$k$ - konstanta tkiva, ki nam pove kako hitro se nasiti tkivo. }
\item{$t$ - čas}
\end{itemize}
Dobili smo nehomogeno diferencialno enačbo prvega reda z konstatnimi koeficienti. 
\begin{equation}
\frac{\mathrm{d} P(t)}{\mathrm{d} t}+k P(t)=k P_{alv}(t)
\label{eq:neh1}
\end{equation}
Za reševanje diferencialne enačbe prvega reda v matematiki obstaja naslednji recept. Najprej rešimo homogeni del enačbe. 
\begin{equation}
\frac{\mathrm{d} P(t)}{\mathrm{d} t}+k P(t)=0 \quad \frac{\mathrm{d} P(t)}{P} = -k\mathrm{d}t
\end{equation} 
Po integraciji dobimo naslednjo enačbo
\begin{equation}
\ln P(t)=-kt+\ln C' \quad P(t) = Ce^{-kt},
\end{equation} 
pri čemer sta $C$ in $C'$ konstanti. Rešitev nehomogene enačbe \ref{eq:neh1} dobimo z variacijo konstante.
Vstavimo v enačbo in dobimo
\begin{equation}
C'e^{-kt}-Ce^{-kt}+Ce^{-kt}=C'e^{-kt}=kP_{alv}
\end{equation} 
Pri tem je $C'=\frac{\mathrm{d} C}{\mathrm{d} t}$. Po integraciji po času dobimo
\begin{equation}
C=P_{alv}e^{kt}+D
\end{equation}
Pri čemer je $D$ nova konstanta, saj izvedli nedoločen integral. 
Splošna rešitev se sedaj glasi 
\begin{equation}
P(t)=P_{alv}+De^{-kt}
\end{equation}
Za določitev konstante D upoštevamo še robne pogoje. Pri času $t$=0 upoštevamo da je tkivo že nasišeno z nekim parcialnim tlakom $P_{t_0}$.
\begin{equation} 
P(0)=P_{t_0}=P_{alv}+D \quad D=P_{t_0}-P_{alv}
\end{equation}
Vstavimo kostatno $D$ v enačbo  in končna rešitev se glasi
\begin{equation}
P(t)=P_{alv}+[P_{t_0}-P_{alv}]e^{-kt}
\end{equation}
Enačbo lahko nekoliko preoblikujemo in dobimo 
\begin{equation}
P(t)=P_{t_0}+[P_{alv}-P_{t_0}][1-e^{-kt}],
\end{equation}
ki je v dekompresijski literaturi znana kot Haldanova enačba. 
\section{Absorbcija inertnega plina v tkivo pri dvigu oziroma spustu}
Pri potopu potapljač tudi spreminja globino. V tem poglavju si bomo pogledali kako se spreminja tlak inertnega plina pri dvigovanju ali spučanju s konstantno hitrostjo. To pomeni da se parcialni tlak dihanega plina spreminja linearno s časom. Spreminjanje tlaka $P_{alv}$  zapišemo kot
\begin{equation}
P_{alv}=P_{alv_0}+rt
\end{equation}
Pri tem je $P_{alv0}$ začetni parcialni tlak inertnega plina pri času $t$=0 in $r$ hitrost dvigovanja v enotah tlaka na časovno enoto. Količina $r$ je negativna pri dvigovanju in pozitivna pri spučanju v globino. 
Enačba se zapiše kot
\begin{equation}
  \frac{\mathrm{d} P(t)}{\mathrm{d} t}=k\left(P_{alv_0}+rt-P(t)\right) \quad  \frac{\mathrm{d} P(t)}{\mathrm{d} t} + kP(t) = kP_{alv_0}+krt 
\end{equation}   
Zopet najprej rešujemo homogeni del enačbe. 
\begin{equation}
\frac{\mathrm{d} P(t)}{\mathrm{d} t} + kP(t)=0 \quad \frac{\mathrm{d} P(t)}{P(t)} = -k\mathrm{d} t 
\end{equation}
Po integraciji dobimo
\begin{equation}
\ln P(t) = -kt + \ln C \quad P(t) = Ce^{-kt}
\end{equation}
Zopet uporabimo metodo variacije konstante. Vstavimo v enačbo in dobimo
\begin{equation}
C' e^{-kt}-kC e^{-kt}+kC e^{-kt}=kP_{alv_0}+krt \quad C'=\frac{\mathrm{d} C}{\mathrm{d} t}=kP_{alv_0}e^{kt}+krt e^{kt}
\end{equation}
Enačbo sedaj integriramo po času in dobimo
\begin{equation}
C=P_{alv_0} e^{kt}+rt e^{kt}-\frac{r}{k}e^{kt}+D
\end{equation}
Vstavimo v enačbo in dobimo
\begin{equation}
P(t)=P_{alv_0}+rt-\frac{r}{k}+De^{-kt}
\end{equation}
Za določitev konstante $D$ je potrebno upoštevati robni pogoj, ki se glasi $P(t=0)=P_{t_0}$. Iz tega pogoja za konstatno $D$ dobimo
\begin{equation}
D=P_{t_0}-P_{alv_0}+\frac{r}{k}
\end{equation}
Končna rešitev se glasi
\begin{equation}
P(t)=P_{alv_0}+r(t-\frac{1}{k})+[P_{t_0}-P_{alv_0}+\frac{r}{k}]e^{-kt}
\end{equation}
Enačba je v dekompresijski literaturi znana kot Schreinerjeva enačba, saj je do nje prvi prišel Schreiner.


\end{document}












